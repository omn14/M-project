%This is chapter 1
%%=========================================
\chapter{Introduction}
The first chapter of a well-structured thesis is always an introduction, setting the scene with background, problem description, objectives, limitations, and then looking ahead to summarize what is in the rest of the report. This is the part that readers look at first---\emph{so make sure it hooks them!}

%%=========================================
\section{Background}
Soil is a complex medium.
Its inhomogenous nature means that the physical parameters of soil vary spatialy both vertically and lateraly.
Traditionally soil properties is modeled with a representative value, usually some kind of mean value or simmilar.
In probabilistic methods, this variability, or uncertainty is taken into account by treating the soil as a random variable sampled from a probability distribution.
By using random field theory and statistics one can try to describe how the soil parameters are distributed in space and how they vary with distance.
In a slope stability problem, the spatial distribution of the soil strength governing parameters has a direct impact on the development of the failure surface, the falure mechanism and therfore the over all stability of the slope.
To simulate stability a finite element program can be used with the soil model parameters input to the finite element mesh based on statistical spatial correlated random fields.
To simulate variability and uncertainty, the modeling is repeated many times with different random fields.
This is the random finite element method.


In this section, you should present the problem that you are going to investigate or analyze; why this problem is of interest; what has, so far, been done to solve the problem, and which parts of the problem that remain.
%%=========================================
\subsection*{Problem Formulation}
Current modern soil modeling software do not support random finite element method, and published reasearch random finite element software code do not have the advanced functionality as modern comercial geotechnical software.
How ever, since the random finite element method does not change the way the problem is simulated, only the input parameters change, modern software packages can be used if a way to specify the input and the simulation run parameters can be controlled in an efficient manner.
Plaxis, a modern software package devolped by Bently, has capabilities like this through its application program interface and python scripting. 
It is of great interest to research on spatialy varying soil modeling to utelize the anvanced functions of existing software. Gaining this ability will expand the compexity of the problems alowed to be simulated by the RFEM method.


You should define your problem in a clear an unambiguous way and explain why this is a problem, why it is of interest---and to whom. It is also important to delimit the problem area.
%%=========================================
\subsection*{Literature Survey}
You should here present the main books and articles that treat problems that are similar to what  you are studying. If you,  later in your thesis, describe the ``state of the art'' -- with a detailed literature survey, you may just give a very brief survey here (approx. a quarter of a page). If this is the only literature survey, you need to go into more details. An objective of the literature survey is to show the reader that you are familiar with the main literature within your field of research -- so that you do not ``reinvent the wheel.''


References to literature can be given in two different ways:
\begin{itemize}
\item As an \emph{explicit} reference: It is shown by \citet{gri2010NUM} and partly also by \citet{Emd2007}  that \ldots.
\item As an \emph{implicit} reference: It is shown \citep[e.g., see][Chap. 4]{Deg2011Geo} that \ldots.
\end{itemize}
In the example above, we have used ``author-year'' references, which is the preferred format. It might be wise to use a program like EndNote to keep track on the references. NTNU has licenses for EndNote. When you refer to the scientific literature, you should always write in \emph{present} tense. Example: \citet{gri2010NUM} show that \ldots. 
\begin{remark}
 You may include a link to the Internet in the text by using a command like: \url{http://www.ntnu.edu/}.
\end{remark} 


%%=========================================
\subsection*{What Remains to be Done?}
After you have defined and delimited your problem -- and presented the relevant results found in the literature within this field, you should sum up which parts of the problem that remain to be solved.
%%=========================================
\section{Objectives}
The main objectives of this project are
\begin{enumerate}
\item This is the first objective
\item This is the second objective
\item This is the third objective
\item More objectives
\end{enumerate}

All objectives shall be stated such that we, after having read the thesis, can see whether or not you have met the objective. ``To become familiar with \ldots'' is therefore not a suitable objective.

%%=========================================
\section{Limitations}
In this section you describe the limitations of your study. These may be related to the study object (physical limitations, operational limitations), to the thoroughness of the analysis, and so on.
%%=========================================
\section{Approach}
Here you should describe the (scientific) approach that you will use to solve the problem and meet your objectives. You should specify the approach for each objective.

If there are any ethical problems related to your approach, these should be highlighted and discussed.
%%=========================================
\section{Structure of the Report}
The rest of the report is structured as follows. Chapter 2 gives an introduction to \ldots

\begin{remark}
Notice that chapter and section headings shall be written in lowercase, but that all main words should start with a capital letter.
\end{remark}
\vspace{2pc}

The length of the report is not important, \underline{the content is}! 
