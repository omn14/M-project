%This is chapter 2
%%=========================================
\chapter[Theory]{Theory}
The content of this chapter will vary with the topic of your thesis. 

\section{Random Fields}

To model the spatial variability of soils random fields are used. To describe the random field of a soil parameter, e. g. the undrained shear strength of the soil, three parameters are commonly used. 
The Mean, $\mu$, Standart deviation, $\sigma$ of the soil parameters underlying probability distribution and the Spatial correlation length, $\theta$ also known as the scale of fluctuation.   
The mean is a measure of around which value the soil strength parameter is distributed. The standard deviation tells how much the values in the soil random fields varies in values. The spatial correlation length, or scale of fluctuation, is a measure of how similar in strength points in the spatial random field are depending on how far away the points are from each other. Large spatial correlation length vary smoothly and varying slowly, while scale of fluctuation is jagged and rappidly varying.
Soil samples taken close together is more likely to be similar than soil samples taken far apart.
Du to the process of soil deposition the soil tend to have different properties in differing direction. We say that soil is anisotropic. Typically for layered soils, the soil properties are more similar in the horizontal direction than in the vertical direction. This anisotropy can be represented by having a different scale of fluctuation in the horozontal directial than in the vertical direction.
\remark{Difference between homogenous media and isotropic}

\subsection{SRM - Spectral Representation Method}

Various methods exist for generating random fields that can be used to represent spatialy variable soil. \citep*[see e. g.][Chapter 6]{fenton2008risk}. The methods differ in their efficiency and accuracy and complexity e. g. ability to describe anisotropy.
In this project work the spectral representation method is used. It is showed by \citet*{shinozuka1996simulation} how to simulate multi-dimensional homogeneous stochastic fields using the spectral representation method. 
Sample functions of the stochastic field can be generated using a cosine series formula. These sample functions accurately reflect the prescribed probabilistic characteristics of the stochastic field when the number of terms in the cosine series is large.

\begin{equation}
\label{eq114}
	f(x,y) = \sqrt{2}\sum_{n_1=0}^{N_1-1} \sum_{n_2=0}^{N_2-1} A_{n_1 n_2} [cos(\kappa_{1_{n_1}} x_1 + \kappa_{2_{n_2}} x_2 + \Phi_{n_1 n_2}^{(1)}) + cos(\kappa_{1_{n_1}} x_1 - \kappa_{2_{n_2}} x_2 + \Phi_{n_1 n_2}^{(2)}) ]
\end{equation}

where:

\begin{equation}
\label{eq45}
	A_{n_1 n_2} = \sqrt{2S_{f_0 f_0}(\kappa_{1_{n_1}},\kappa_{2_{n_2}})\Delta\kappa_1\Delta\kappa_2}
\end{equation}

\begin{equation}
\label{eq47}
	\kappa_{1_{n_1}} = n_1 \Delta\kappa_1 \quad\text{and}\quad \kappa_{2_{n_2}} = n_2 \Delta\kappa_2
\end{equation}

\begin{equation}
\label{eq48}
	\Delta\kappa_1 = \frac{\kappa_{1u}}{N_1} \quad\text{and}\quad \Delta\kappa_2 = \frac{\kappa_{2u}}{N_2} 
\end{equation}

and:

\begin{equation}
\label{eq49}
	A_{{0}n_2} = A_{{n_1}0} = 0 \quad\text{for}\quad n_1 = n_2 = 0,1,\dots,N_1-1
\end{equation}



See chapter 3 for details on implementation.

\section{Finite Element Method}

The Finite Element method is a numerical method that can be used to solve a multitude of geotechnical and other engineering problems. 
Being a numerical method, FEM gives approximate solution, care sould be used when constructing the model to be calculated. Things to keep in mind when running FEM analysis is the volume extent and the boundry interface of the model, element type, density of mesh and criteria of convergence.
In comparison to limit equilibrium methods (LEM), FEM methods can give deformations in a state before ultimate limit state i. e. before failure. 
FEM have no prior assumtions of the falure mechanism i.e. no assumption of a circular falure surface in a LEM slope stability analysis. It is a very attractive property of FEM analysis combined with spatial variable soil, to allow slope falure to develop naturally by finding the path of weakest soil.

The main principles of the FEM is to devide the soil into smaller elements, then develop description of what happening in each element, then add the behaviour of each element together to find the behaviour of the whole volume.

Seven steps of a finite element program is described by \citet{nordal2020compendium}:
\begin{enumerate}
	\item{Element modeling, equations are built for each element, integration in points inside the elemnt is used to form stiffness matix}
	\item{Global modelling, all element siffness matrixes are assembled into a system of equations to form a global stiffness matrix. An incremental load vector is found.}
	\item{Equation solving of the global equation system for the load increment. This gives a displacement increment}
	\item{Stress evaluation of the calculated displacement increment. The displacments are used to find the strains wich is used to find the stress increments}
	\item{Testing for numerical accuracy. If the calculations show too high unbalanced forces, it will be necessary to adjust the load increment and/or add more iterations. if so step 1-5 must be recalculated. When the results converges the program proceeds to step 6.}
	\item{Updating of results by adding the deformations and stress to form total deformations and total stresses.}
	\item{Calculation of new load increment. The response of the new load increment is found by repeating 1 to 6. New load increment is repeatedly added until the specified external load is reached, or failure occur.}
\end{enumerate}
\remark{Plaxis measures the fraction of the applied loadstep according to step 7 above in a variabel wich in this project is accessed to automatically tell if we have faliure or not.}



